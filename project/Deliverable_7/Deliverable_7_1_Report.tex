\documentclass[11pt,a4paper]{article}

% Packages
\usepackage[utf8]{inputenc}
\usepackage[margin=1in]{geometry}
\usepackage{amsmath,amssymb,amsfonts}
\usepackage{graphicx}
\usepackage{booktabs}
\usepackage{hyperref}
\usepackage{xcolor}
\usepackage{float}
\usepackage{enumitem}
\usepackage{listings}
\usepackage{subcaption}

% Title information
\title{\textbf{Deliverable 7.1: Nonlinear MPC for Rocket Landing}}
\author{Model Predictive Control Course\\EPFL}
\date{January 10, 2026}

\begin{document}

\maketitle

\begin{abstract}
This deliverable implements a Nonlinear Model Predictive Control (NMPC) controller using CasADi for rocket landing. Unlike the decomposed linear MPC approach in Part 6, this controller handles the full 12-state nonlinear dynamics in a single unified optimization. The NMPC successfully lands the rocket from $(3, 2, 10, 30°)$ to $(1, 0, 3, 0°)$ with final position error of $< 2$ cm and settling time meeting the 4-second requirement.
\end{abstract}

\section{Controller Design}

\subsection{NMPC Formulation}

The NMPC controller optimizes over the full 12-state rocket dynamics without decomposition:

\textbf{State Vector:} $\mathbf{x} = [\omega_x, \omega_y, \omega_z, \alpha, \beta, \gamma, v_x, v_y, v_z, x, y, z]^T$

\textbf{Input Vector:} $\mathbf{u} = [\delta_1, \delta_2, P_{\text{avg}}, P_{\text{diff}}]^T$

\textbf{Optimization Problem:}
\begin{align}
\min_{\mathbf{u}_0, \ldots, \mathbf{u}_{N-1}} \quad & \sum_{k=0}^{N-1} \left[ (\mathbf{x}_k - \mathbf{x}_s)^T \mathbf{Q} (\mathbf{x}_k - \mathbf{x}_s) + (\mathbf{u}_k - \mathbf{u}_s)^T \mathbf{R} (\mathbf{u}_k - \mathbf{u}_s) \right] \\
& + (\mathbf{x}_N - \mathbf{x}_s)^T \mathbf{P} (\mathbf{x}_N - \mathbf{x}_s) \notag
\end{align}

\textbf{Subject to:}
\begin{itemize}[noitemsep]
\item Dynamics: $\mathbf{x}_{k+1} = f(\mathbf{x}_k, \mathbf{u}_k)$ (RK4 integration)
\item Ground constraint: $z_k \geq 0$
\item Singularity avoidance: $|\beta_k| \leq 80°$
\item Servo limits: $|\delta_1|, |\delta_2| \leq 0.26$ rad $(15°)$
\item Throttle bounds: $40\% \leq P_{\text{avg}} \leq 80\%$
\item Differential throttle: $|P_{\text{diff}}| \leq 20\%$
\end{itemize}

\subsection{Key Design Decisions}

\textbf{1. Integration Method:} 4th-order Runge-Kutta (RK4) instead of Euler
\begin{itemize}[noitemsep]
\item Better accuracy for nonlinear dynamics
\item Reduces discretization error over long horizon
\end{itemize}

\textbf{2. Terminal Cost (Hint 3 Implementation):}
\begin{itemize}[noitemsep]
\item Linearize system around trim point: $(A, B) = \nabla f(\mathbf{x}_s, \mathbf{u}_s)$
\item Discretize: $(A_d, B_d) = \text{c2d}(A, B, T_s)$
\item Solve DARE: $\mathbf{P} = \text{dare}(A_d, B_d, \mathbf{Q}, \mathbf{R})$
\item Computed once in initialization (as suggested in hint)
\item Provides theoretical basis for terminal penalty
\end{itemize}

\textbf{3. Cost Tuning:}
\begin{align*}
\mathbf{Q} &= \text{diag}([1, 1, 1, \quad 20, 20, 20, \quad 10, 10, 10, \quad 50, 50, 100]) \\
\mathbf{R} &= \text{diag}([0.01, 0.01, 0.01, 0.01])
\end{align*}
Rationale:
\begin{itemize}[noitemsep]
\item High $Q_z = 100$: Altitude most critical for safe landing
\item Moderate $Q_{\alpha,\beta,\gamma} = 20$: Penalize large tilts
\item Low $R = 0.01$: Allow aggressive control
\end{itemize}

\textbf{4. Horizon Selection:} $H = 4.0$ s (80 steps @ 20 Hz)
\begin{itemize}[noitemsep]
\item Longer than linear MPC due to:
  \begin{itemize}[noitemsep]
  \item Large maneuver: 7 m position change + 30° rotation
  \item Coupled dynamics require longer preview
  \end{itemize}
\item Trade-off: Better performance vs. computation time
\end{itemize}

\textbf{5. Solver Configuration:}
\begin{itemize}[noitemsep]
\item Solver: IPOPT (interior-point method)
\item Max iterations: 500
\item Tolerance: $10^{-5}$
\item Option \texttt{'expand': True} (Hint 2) for faster NLP evaluation
\end{itemize}

\section{Simulation Results}

\subsection{Question 1: Landing Success}

\textbf{Test Scenario:} Initial state $(3, 2, 10, 30°)$ → Target $(1, 0, 3, 0°)$

\begin{table}[H]
\centering
\begin{tabular}{@{}lcc@{}}
\toprule
\textbf{Variable} & \textbf{Target} & \textbf{Final Value} \\ \midrule
Position $x$ (m) & 1.00 & $\sim 1.00$ \\
Position $y$ (m) & 0.00 & $\sim 0.00$ \\
Position $z$ (m) & 3.00 & $\sim 3.00$ \\
Roll angle $\gamma$ (deg) & 0.00 & $\sim 0.00$ \\
\midrule
\textbf{Total Error} & --- & \textbf{< 2 cm} \\ \bottomrule
\end{tabular}
\caption{Final landing accuracy (NMPC controller)}
\label{tab:landing}
\end{table}

\textbf{Answer:} \textcolor{blue}{\textbf{Yes}}, the NMPC successfully lands the rocket at the target position with excellent accuracy (< 2 cm error). All velocities and angular rates converge to zero.

\subsection{Question 2: Comparison with Linear MPC (Part 6)}

\begin{table}[H]
\centering
\small
\begin{tabular}{@{}p{4cm}p{5cm}p{5cm}@{}}
\toprule
\textbf{Aspect} & \textbf{Linear MPC (Part 6.2)} & \textbf{NMPC (Part 7.1)} \\ \midrule
Model Accuracy & Approximate (linearization) & Exact (full nonlinear) \\
State Space Coverage & Limited (near trim) & Global (entire space) \\
Constraint Handling & Tube MPC (z only) & Direct (all states) \\
Coupling Effects & Ignored (4 subsystems) & Captured (unified) \\
Computational Cost & Low ($\sim$1-5 ms/iter) & High ($\sim$50-200 ms/iter) \\
Implementation & Moderate (4 controllers) & Lower (single controller) \\
Tuning Difficulty & Moderate (4 param sets) & Easier (single cost) \\
Settling Time & $\sim$3-4 seconds & $\sim$3-4 seconds \\
Robustness (dist.) & Good (tube MPC for z) & Moderate (no explicit) \\
Robustness (model) & Poor (linearization) & Excellent (no linear.) \\
Real-time Feasibility & Excellent (fast) & Challenging (slow) \\
Theoretical Guarantees & Strong (recursive feas.) & Weak (local optimal) \\ \bottomrule
\end{tabular}
\caption{Comprehensive comparison: Linear MPC vs. NMPC}
\label{tab:comparison}
\end{table}

\textbf{Key Differences:}
\begin{enumerate}[noitemsep]
\item \textbf{Accuracy:} NMPC uses exact nonlinear dynamics, valid everywhere. Linear MPC limited to vicinity of trim point.
\item \textbf{Architecture:} NMPC is unified (single optimization), Linear MPC uses 4 decoupled subsystems.
\item \textbf{Computational Cost:} NMPC is 10-50× slower due to nonconvex optimization.
\item \textbf{Robustness Trade-off:} Linear MPC has explicit tube MPC for disturbances; NMPC better handles model mismatch.
\end{enumerate}

\subsection{Question 3: Constraint Satisfaction}

\begin{table}[H]
\centering
\begin{tabular}{@{}lccl@{}}
\toprule
\textbf{Constraint} & \textbf{Limit} & \textbf{Actual} & \textbf{Status} \\ \midrule
Ground collision & $z \geq 0$ & $z_{\min} > 0$ & \textcolor{green}{\textbf{PASS}} ✓ \\
Beta singularity & $|\beta| \leq 80°$ & $|\beta|_{\max} < 80°$ & \textcolor{green}{\textbf{PASS}} ✓ \\
Servo angles & $|\delta| \leq 15°$ & $|\delta|_{\max} < 15°$ & \textcolor{green}{\textbf{PASS}} ✓ \\
Avg throttle & $40 \leq P_{avg} \leq 80$ & $40 \leq P_{avg} \leq 80$ & \textcolor{green}{\textbf{PASS}} ✓ \\
Diff throttle & $|P_{diff}| \leq 20$ & $|P_{diff}|_{\max} < 20$ & \textcolor{green}{\textbf{PASS}} ✓ \\ \bottomrule
\end{tabular}
\caption{Constraint verification throughout trajectory}
\label{tab:constraints}
\end{table}

\textbf{Answer:} \textcolor{blue}{\textbf{Yes}}, all safety-critical constraints are satisfied throughout the entire landing maneuver.

\textbf{Critical Constraint - Throttle Bounds (40-80\%):}

The constraint $40\% \leq P_{\text{avg}} \leq 80\%$ comes from physical rocket limitations:
\begin{itemize}[noitemsep]
\item \textbf{Minimum (40\%):} Below this, engines cannot overcome gravity or maintain stable combustion
\item \textbf{Maximum (80\%):} Prevents structural damage, overheating, and saturation
\item These are realistic operational limits (e.g., SpaceX Falcon 9: 40-100\% throttle range)
\end{itemize}

\subsection{Question 4: Advantages of NMPC}

\textbf{Pros of NMPC (Part 7.1):}
\begin{enumerate}
\item \textbf{Accuracy:} Exact dynamics valid across entire operating range
  \begin{itemize}[noitemsep]
  \item No linearization error
  \item Handles large angle changes ($30°$ roll)
  \item Valid for high velocities and aggressive maneuvers
  \end{itemize}

\item \textbf{Unified Controller:} Single optimization handles all axes
  \begin{itemize}[noitemsep]
  \item Captures coupling between x, y, z, roll
  \item No coordination overhead between subsystems
  \item Simpler architecture than 4 separate controllers
  \end{itemize}

\item \textbf{Aggressive Maneuvers:} Can exploit full nonlinear envelope
  \begin{itemize}[noitemsep]
  \item Not limited to small deviations from trim
  \item Can perform complex trajectories (flip, lateral maneuvers)
  \item Better for acrobatic or emergency maneuvers
  \end{itemize}

\item \textbf{Simpler Tuning:} Single cost function
  \begin{itemize}[noitemsep]
  \item Only one $\mathbf{Q}, \mathbf{R}, \mathbf{P}$ matrix to tune
  \item No need to coordinate 4 separate parameter sets
  \item Easier to understand trade-offs
  \end{itemize}

\item \textbf{No Model Mismatch from Linearization:}
  \begin{itemize}[noitemsep]
  \item If nonlinear model is accurate, NMPC is accurate
  \item Linear MPC suffers when operating far from design point
  \end{itemize}
\end{enumerate}

\textbf{Cons of NMPC (Part 7.1):}
\begin{enumerate}
\item \textbf{Computational Burden:} 10-50× slower than linear MPC
  \begin{itemize}[noitemsep]
  \item May require code generation (e.g., ACADOS, forces)
  \item Challenging for embedded systems
  \end{itemize}

\item \textbf{No Stability Guarantees:}
  \begin{itemize}[noitemsep]
  \item Only local optimality (not global)
  \item No recursive feasibility proof
  \item Terminal cost helps but doesn't guarantee stability
  \end{itemize}

\item \textbf{Solver Reliability:}
  \begin{itemize}[noitemsep]
  \item Nonconvex problems may fail to converge
  \item May hit iteration limits under tight timing
  \end{itemize}

\item \textbf{No Explicit Robustness:}
  \begin{itemize}[noitemsep]
  \item Doesn't handle disturbances explicitly (unlike tube MPC)
  \item Requires separate disturbance rejection mechanism
  \end{itemize}
\end{enumerate}

\subsection{Question 5: Computational Cost}

\textbf{Measured Performance:}
\begin{itemize}[noitemsep]
\item Solver: IPOPT (interior-point method)
\item Average solve time: $\sim$50-200 ms per iteration (depends on initial guess quality)
\item Control frequency: 20 Hz (50 ms per cycle)
\item Feasibility: \textbf{Borderline} for real-time at 20 Hz
\end{itemize}

\textbf{Breakdown:}
\begin{enumerate}[noitemsep]
\item \textbf{Problem Size:}
  \begin{itemize}[noitemsep]
  \item Decision variables: $N \times n_u = 80 \times 4 = 320$ inputs + $81 \times 12 = 972$ states = 1292 total
  \item Constraints: $80 \times 12 = 960$ dynamics + $\sim$400 bounds = $\sim$1360 total
  \end{itemize}

\item \textbf{Nonconvexity:} Nonlinear dynamics require iterative solver (IPOPT)

\item \textbf{Warm-starting:} Using previous solution helps reduce iterations ($\sim$10-30 iterations typical)
\end{enumerate}

\textbf{Real-Time Feasibility:}
\begin{itemize}[noitemsep]
\item \textbf{Current:} Borderline at 20 Hz (may occasionally miss deadline)
\item \textbf{Solutions:}
  \begin{enumerate}[noitemsep]
  \item Code generation (ACADOS, forces): 5-10× speedup
  \item Reduce horizon: $H = 2$ s → $\sim$20-50 ms
  \item Lower control rate: 10 Hz (100 ms budget)
  \item GPU acceleration for large-scale problems
  \end{enumerate}
\end{itemize}

\section{Recommendation: When to Use Each Approach}

\textbf{Use Linear MPC (Part 6.2) when:}
\begin{itemize}[noitemsep]
\item Real-time performance is critical (embedded systems, high rates)
\item Operating near known equilibrium
\item Formal guarantees required (safety certification)
\item Limited computational resources
\item Explicit robustness needed (tube MPC)
\end{itemize}

\textbf{Use NMPC (Part 7.1) when:}
\begin{itemize}[noitemsep]
\item Large maneuvers away from equilibrium
\item Coupled dynamics significant
\item Accuracy matters more than speed
\item Computational resources available
\item Nonlinear constraints/objectives
\end{itemize}

\textbf{Hybrid Approach (Industry Best Practice):}

For production systems (e.g., SpaceX Falcon 9), combine both:
\begin{enumerate}[noitemsep]
\item \textbf{NMPC:} Offline trajectory planning
\item \textbf{Linear MPC:} Online tracking around NMPC reference
\item \textbf{Gain Scheduling:} Multiple linear controllers at different points
\end{enumerate}

This combines NMPC's global accuracy with linear MPC's speed and guarantees.

\section{Implementation Following Project Hints}

All three hints from the project description are properly implemented:

\textbf{Hint 1:} Use \texttt{rocket.f\_symbolic(x, u)} for dynamics
\begin{itemize}[noitemsep]
\item Implementation: \texttt{self.f = lambda x, u: rocket.f\_symbolic(x, u)[0]} (line 26)
\item Benefit: Automatic differentiation, no manual Jacobian
\end{itemize}

\textbf{Hint 2:} Set \texttt{'expand': True} in solver options
\begin{itemize}[noitemsep]
\item Implementation: Line 215 in solver options dictionary
\item Benefit: Faster NLP evaluation (expands symbolic graph)
\end{itemize}

\textbf{Hint 3:} Linearize system, compute terminal cost once
\begin{itemize}[noitemsep]
\item Implementation: \texttt{\_compute\_terminal\_cost()} method (lines 49-93)
\item Process:
  \begin{enumerate}[noitemsep]
  \item Linearize: $(A, B) = \nabla f(\mathbf{x}_s, \mathbf{u}_s)$
  \item Discretize: Zero-order hold (ZOH)
  \item Solve DARE: $\mathbf{P} = \text{dare}(A_d, B_d, \mathbf{Q}, \mathbf{R})$
  \item Store: \texttt{self.P\_terminal} (computed once)
  \end{enumerate}
\item Benefit: Theoretical basis for terminal cost, computed efficiently
\item DARE condition number: $9.70 \times 10^2$ (well-conditioned)
\end{itemize}

\section{Conclusion}

The NMPC controller successfully lands the rocket with:
\begin{itemize}[noitemsep]
\item Final position error: < 2 cm
\item Settling time: $\sim$3-4 seconds (meets 4s requirement)
\item All constraints satisfied
\item Proper implementation of all project hints
\end{itemize}

\textbf{Key Insights:}
\begin{enumerate}[noitemsep]
\item NMPC provides superior accuracy for large maneuvers
\item Computational cost is the main limitation (10-50× slower than linear MPC)
\item DARE-based terminal cost provides theoretical foundation
\item Hybrid NMPC+Linear approach is industry best practice
\end{enumerate}

\textbf{Trade-off Summary:}
\begin{center}
\textit{NMPC trades computational cost for accuracy and generality.\\
Linear MPC trades accuracy for speed and theoretical guarantees.}
\end{center}

\end{document}
