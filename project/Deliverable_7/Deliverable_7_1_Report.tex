\documentclass[11pt,a4paper]{article}

% Packages
\usepackage[utf8]{inputenc}
\usepackage[margin=1in]{geometry}
\usepackage{amsmath,amssymb,amsfonts}
\usepackage{graphicx}
\usepackage{booktabs}
\usepackage{hyperref}
\usepackage{xcolor}
\usepackage{float}
\usepackage{enumitem}
\usepackage{listings}
\usepackage{subcaption}

% Title information
\title{\textbf{Deliverable 7.1: Nonlinear MPC for Rocket Landing}}
\author{Model Predictive Control Course\\EPFL}
\date{January 10, 2026}

\begin{document}

\maketitle

\begin{abstract}
This deliverable implements a Nonlinear MPC (NMPC) controller using CasADi for rocket landing. The controller handles the full 12-state nonlinear dynamics in a unified optimization. NMPC successfully lands the rocket from $(3, 2, 10, 30°)$ to $(1, 0, 3, 0°)$ with $< 2$ cm error and settling time $< 4$ seconds.
\end{abstract}

\section{Controller Design}

\subsection{NMPC Formulation}

The NMPC controller optimizes over the full 12-state rocket dynamics without decomposition:

\textbf{State Vector:} $\mathbf{x} = [\omega_x, \omega_y, \omega_z, \alpha, \beta, \gamma, v_x, v_y, v_z, x, y, z]^T$

\textbf{Input Vector:} $\mathbf{u} = [\delta_1, \delta_2, P_{\text{avg}}, P_{\text{diff}}]^T$

\textbf{Optimization Problem:}
\begin{align}
\min_{\mathbf{u}_0, \ldots, \mathbf{u}_{N-1}} \quad & \sum_{k=0}^{N-1} \left[ (\mathbf{x}_k - \mathbf{x}_s)^T \mathbf{Q} (\mathbf{x}_k - \mathbf{x}_s) + (\mathbf{u}_k - \mathbf{u}_s)^T \mathbf{R} (\mathbf{u}_k - \mathbf{u}_s) \right] \\
& + (\mathbf{x}_N - \mathbf{x}_s)^T \mathbf{P} (\mathbf{x}_N - \mathbf{x}_s) \notag
\end{align}

\textbf{Subject to:} RK4-integrated dynamics $\mathbf{x}_{k+1} = f(\mathbf{x}_k, \mathbf{u}_k)$, ground safety $z_k \geq 0$, singularity avoidance $|\beta_k| \leq 80°$, servo limits $|\delta_1|, |\delta_2| \leq 0.26$ rad, throttle bounds $40\% \leq P_{\text{avg}} \leq 80\%$, and differential throttle $|P_{\text{diff}}| \leq 20\%$.

\subsection{Key Design Decisions}

\textbf{1. Integration Method:} 4th-order Runge-Kutta (RK4) provides better accuracy for nonlinear dynamics and reduces discretization error over the long horizon.

\textbf{2. Terminal Cost (Hint 3 Implementation):} The system is linearized around the trim point $(A, B) = \nabla f(\mathbf{x}_s, \mathbf{u}_s)$, discretized via $(A_d, B_d) = \text{c2d}(A, B, T_s)$, and the DARE solution $\mathbf{P} = \text{dare}(A_d, B_d, \mathbf{Q}, \mathbf{R})$ is computed once during initialization to provide a theoretical basis for the terminal penalty.

\textbf{3. Cost Tuning:}
\begin{align*}
\mathbf{Q} &= \text{diag}([1, 1, 1, \quad 20, 20, 20, \quad 10, 10, 10, \quad 50, 50, 100]) \\
\mathbf{R} &= \text{diag}([0.01, 0.01, 0.01, 0.01])
\end{align*}
High altitude weight $Q_z = 100$ prioritizes safe landing, moderate angle weights $Q_{\alpha,\beta,\gamma} = 20$ penalize large tilts, and low input penalty $R = 0.01$ allows aggressive control.

\textbf{4. Horizon Selection:} $H = 4.0$ s (80 steps @ 20 Hz) provides sufficient preview for the large maneuver (7 m displacement + 30° rotation) while balancing performance and computation time.

\textbf{5. Solver:} IPOPT with \texttt{'expand':True}, max iterations 500, tolerance $10^{-5}$

\section{Simulation Results}

\subsection{Open-Loop and Closed-Loop Performance}

The controller was tested from initial state $(3, 2, 10, 30°)$ to target $(1, 0, 3, 0°)$.

\begin{figure}[H]
\centering
\includegraphics[width=0.85\textwidth]{figures_deliverable_7_1/01_open_loop_trajectory.png}
\caption{NMPC open-loop optimal trajectory from initial state}
\label{fig:open_loop}
\end{figure}

\subsection{Landing Success}

\textbf{Result:} The rocket successfully lands with final position error of \textbf{1.92 cm} from target $(1, 0, 3)$ m.

\subsection{Comparison with Linear MPC}

\begin{figure}[H]
\centering
\begin{subfigure}[b]{0.48\textwidth}
\centering
\includegraphics[width=\textwidth]{comparison_figures/position_comparison.png}
\caption{Position trajectories}
\label{fig:position}
\end{subfigure}
\hfill
\begin{subfigure}[b]{0.48\textwidth}
\centering
\includegraphics[width=\textwidth]{comparison_figures/attitude_comparison.png}
\caption{Attitude evolution}
\label{fig:attitude}
\end{subfigure}
\caption{Trajectory comparison: Linear MPC (blue) vs NMPC (red)}
\label{fig:comparison}
\end{figure}

\begin{table}[H]
\centering
\begin{tabular}{lcc}
\toprule
Metric & Linear MPC (6.2) & NMPC (7.1) \\
\midrule
Final Position Error & \textbf{0.06 cm} & 1.92 cm \\
Settling Time (4s req.) & 5.10 s (FAIL) & \textbf{$<$ 4 s (PASS)} \\
Computation Time/Step & \textbf{25.64 ms} & 50-200 ms \\
Constraint Violations & 0 & 0 \\
\bottomrule
\end{tabular}
\caption{Performance comparison: Linear MPC vs NMPC}
\label{tab:comparison}
\end{table}

\textbf{Analysis:} NMPC meets settling time requirement; Linear MPC does not. NMPC handles nonlinearities directly in unified optimization. Both satisfy all constraints.

\begin{figure}[H]
\centering
\includegraphics[width=0.85\textwidth]{comparison_figures/control_comparison.png}
\caption{Control inputs comparison: Linear MPC (blue) vs NMPC (red)}
\label{fig:control}
\end{figure}

\subsection{Settling Time Verification}

\begin{figure}[H]
\centering
\includegraphics[width=0.85\textwidth]{figures_deliverable_7_1/02_settling_time_verification.png}
\caption{NMPC settling time verification with tolerance bands}
\label{fig:settling}
\end{figure}

From Figure~\ref{fig:settling}, settling time $< 4$ s is verified with position error $< 0.05$ m at 3.5 s, velocity error $< 0.01$ m/s at 4.0 s, and angle error $< 1°$ at 3.8 s.

\subsection{Pros and Cons}

\textbf{Advantages:} NMPC uses exact nonlinear dynamics without linearization error, employs unified optimization that captures coupling between axes, and excels at large maneuvers and aggressive trajectories. \textbf{Disadvantages:} Higher computational cost (4-8× slower than linear MPC), no formal stability guarantees, and potential convergence failures under tight timing constraints.

\begin{figure}[H]
\centering
\includegraphics[width=0.85\textwidth]{figures_deliverable_7_1/03_constraint_verification.png}
\caption{NMPC constraint verification throughout trajectory}
\label{fig:constraints}
\end{figure}

NMPC is recommended when fast settling time is critical, large state deviations are expected, and computational resources are available. Linear MPC is preferred when high precision is required, computation is limited, or operation is near hover conditions.

\section{Conclusion}

The NMPC controller successfully lands the rocket with 1.92 cm error and settling time $< 4$ s. All constraints are satisfied. NMPC provides better settling time than Linear MPC but requires more computation (4-8× slower).

\end{document}
